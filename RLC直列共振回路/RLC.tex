\documentclass[11pt,a4paper,fleqn]{jsarticle}
%
%\usepackage{amsmath,amssymb}
\usepackage{newtxtext, newtxmath}
\usepackage{bm}
\usepackage{graphicx}
\usepackage{ascmac}
\usepackage{subfigure}
\usepackage{siunitx}
\usepackage{url}
\usepackage{here}
%
\setlength{\textwidth}{\fullwidth}
\setlength{\textheight}{40\baselineskip}
\addtolength{\textheight}{\topskip}
\setlength{\voffset}{-0.2in}
\setlength{\topmargin}{0pt}
\setlength{\headheight}{0pt}
\setlength{\headsep}{0pt}
\allowdisplaybreaks
%
\newcommand{\divergence}{\mathrm{div}\,}  %ダイバージェンス
\newcommand{\grad}{\mathrm{grad}\,}  %グラディエント
\newcommand{\rot}{\mathrm{rot}\,}  %ローテーション
%
\title{RLC直列共振回路}
\author{E1534\\根津 宏輔}
\date{\today}
%
\begin{document}
\section{目的}
RLC直列共振回路の特性を理解し、これを実験的に確かめること。
%
\section{原理}
図\ref{fig:1a}にRLC直列回路を示す。
\begin{figure}[htbp]
\label{fig:1a}
\center{
 \includegraphics[clip, width=0.35\columnwidth]{img/1a.eps}
 \caption{RLC直列回路}
}
\end{figure}

いま回路に流れる電流を、$i=\sqrt{2}I\sin \omega t$と仮定すると

\begin{eqnarray}
\left\{\vbox to 24pt{} \begin{array}{ll}
e_{R}&=\sqrt{2} RI\sin \omega t\\
e_{L}&=\sqrt{2}\omega LI \sin \biggl( \omega t+\dfrac{\pi}{2}\biggr)\\
e_{C}&=\sqrt{2}\frac{1}{\omega C}I \sin \biggl( \omega t-\dfrac{\pi}{2}\biggr)\\
e&=e_{R}+e_{L}+e_{C}=\sqrt{2}E\sin (\omega t + \theta)
\end{array} \right.
\end{eqnarray}

これらをベクトル図に示した物が図\ref{fig:1b}である。
\begin{figure}[h]
\label{fig:1b}
\center{
\includegraphics[clip, width=0.35\columnwidth]{img/1b.eps}
\caption{ベクトル図}
}
\end{figure}
これより
\begin{align}
&E^{2}={E_{R}}^{2}+(E_{L}-E{C})^{2}=\Biggl\{R^{2}+\biggl(\omega L-\frac{1}{\omega C}\biggr)^{2}\Biggr\}I^{2}=Z^{2}I^{2}\nonumber\\
&\therefore E=ZI=\Biggl\{R^{2}+\biggl(\omega L-\frac{1}{\omega C}\biggr)^{2}\Biggr\}^{\frac{1}{2}}I
\end{align}
また
\begin{eqnarray}
\theta &=\tan ^{-1}\frac{E_{L}-E{C}}{E_{R}}=\tan ^{-1}\frac{\omega L-\dfrac{1}{\omega C}}{R}
\end{eqnarray}
ところで図\ref{fig:1a}の回路においてリアクタンス成分が0になる条件を直列共振条件という。
このときは
\begin{align}
&\omega _{0}L-\frac{1}{\omega _{0}C}=0\\
&\omega_{0}=\frac{1}{\sqrt{LC}}\\
&Z_{0}=R\\
&I_{0}=\frac{E}{R}\\
&\theta =\tan^{-1}0=0^\circ
\end{align}
が成立し、インピーダンス$Z$は最小に、電流は最大に、また位相角$\theta$は0となる。
また、このとき
\begin{align}
\left\{\vbox to 24pt{} \begin{array}{ll}
E_{R0}&=RI_{0}=R\times\dfrac{E}{R}=E\\
E_{C0}&=\dfrac{1}{\omega_{0}C}I_{0}=\dfrac{E}{\omega_{0}CR}\\
E_{L0}&=\omega_{0}LI_{0}=\omega_{0}L\times\dfrac{E}{R}
\end{array} \right.
\end{align}
となる。
いま電源周波数$f$を変化したときの$I,\ Z,\ \theta,\ $の変化を図\ref{fig:2}に模式的に示す。
図よりわかるように、直列共振回路は特定の周波数成分の信号を取り出すときに使用される。
以上の性質を実験によって確かめることとする。
\begin{figure}[htbp]
 \centering
 \subfigure[電流の変化]{
  \includegraphics[clip, width=0.3\columnwidth]{img/2a.eps}
  \label{fig:2a}}
 \subfigure[インピーダンスの変化]{
  \includegraphics[clip, width=0.3\columnwidth]{img/2b.eps}
  \label{fig:2b}}
 \subfigure[位相の変化]{
  \includegraphics[clip, width=0.3\columnwidth]{img/2c.eps}
  \label{fig:2c}}
 \caption{直列共振回路の特性}
 \label{fig:2}
\end{figure}

%
\section{実験方法}

実験回路を図\ref{fig:3}に示す。
回路素子部分をブレッドボード上に結線し交流電源として発信器を用いる。
発信器の出力側にオシロスコープのCH1側を接続し、抵抗RにCH2側を接続して波形を観測から実験値を読み取る。
また、適宜デジタルマルチメータを用いて各素子の電圧を観測する。
\begin{figure}[h]
\label{fig:3}
\center{
\includegraphics[clip, width=0.35\columnwidth]{img/3.eps}
\caption{実験回路図}
}
\end{figure}

\subsection{実験1 周波数特性}
(1)実験回路を結線して発振器出力を常に一定に保ちながら周波数を500 Hz~100 kHzまで変化させて各素子の電圧を測定し、周波数に対する電流の特性を測定する、
特に、電流が最大となる今日進展はより細かく測定する。
($R=1\ \si{[k\ohm]}, C=47\ \si{[nF]}, 発振器出力=1\ \si{[V]}$)

(2)抵抗値を変えて(1)の実験を行う。
ただし、束帯電圧は抵抗Rのみとする。
\subsection{実験2 静電容量依存性}
周波数を固定して回路の静電容量を変化させ、制電少量に対する電流の特性を測定する。
コンデンサCの値は$0.001\ \si{\micro F}~0.2\ \si{\micro F}$まで変化させて測定を行う。
($f=11\ \si{[kHz]}, 出力=1\ \si{[V]}, R=1\ \si{[k\ohm]},$Lは(1)で用いた物、抵抗の電圧を測定し、$I=|V_{R}|/R$とする。)
\subsection{実験3 $r_{L}$の測定}
コイルの束帯分$r_{L}$をマルチメータを使って測定する。
以後、コイルの抵抗としては、この値を使用する。

\subsection{使用器具}
この実験で使用した器具を表\ref{cal:item}に示す。
\begin{table}[!h]
\centering
\caption{使用器具}
\label{cal:item}
\begin{tabular}{|l|l|l|l|}
\hline
\multicolumn{1}{|c|}{器具名} & メーカ名    & 型番      & シリアルナンバー   \\ \hline
デュアルディスプレイマルチメータ          & TEXIO   & DL-2040 & 13020563   \\ \hline
デュアルディスプレイマルチメータ          & TEXIO   & DL-2040 & 130205538  \\ \hline
発信器                       & KENWOOD & AG-2040 & 6050017    \\ \hline
可変コンデンサ                   & HP      & 4440B   & 1224J04420 \\ \hline
\end{tabular}
\end{table}
%
\section{実験結果}
\subsection{実験1の結果}
実験1(1)、(2)の結果を表\ref{cal:result1(1)}、\ref{cal:result1(2)}に示す。
\begin{table}[!h]
\centering
\caption{実験1(1)の結果}
\label{cal:result1(1)}
\begin{tabular}{|l|l|l|l|}
\hline
周波数 {[}kHz{]} & Rにかかる電圧 {[}V{]} & Cにかかる電圧 {[}V{]} & Lにかかる電圧 {[}V{]} \\ \hline \hline
0.5           & 0.1561          & 0.98            & 0.0177          \\ \hline
0.6           & 0.1852          & 0.955           & 0.0237          \\ \hline
0.8           & 0.2524          & 0.998           & 0.0404          \\ \hline
1             & 0.3244          & 0.995           & 0.0651          \\ \hline
2             & 0.635           & 0.975           & 0.2474          \\ \hline
3             & 0.863           & 0.896           & 0.496           \\ \hline
3.5           & 0.922           & 0.828           & 0.612           \\ \hline
4             & 0.946           & 0.716           & 0.746           \\ \hline
4.5           & 0.941           & 0.663           & 0.796           \\ \hline
5             & 0.897           & 0.552           & 0.87            \\ \hline
6             & 0.825           & 0.4315          & 0.943           \\ \hline
8             & 0.648           & 0.253           & 0.994           \\ \hline
10            & 0.521           & 0.162           & 1.007           \\ \hline
20            & 0.2631          & 0.0417          & 1.016           \\ \hline
30            & 0.1722          & 0.0181          & 1.004           \\ \hline
40            & 0.1284          & 0.0095          & 1.005           \\ \hline
50            & 0.1025          & 0.0045          & 1.004           \\ \hline
60            & 0.0844          & 0.0018          & 1.007           \\ \hline
80            & 0.061           & 0               & 1.011           \\ \hline
100           & 0.0464          & 0               & 1.004           \\ \hline
\end{tabular}
\end{table}

\begin{table}[!h]
\centering
\caption{実験1(2)の結果}
\label{cal:result1(2)}
\begin{tabular}{|l|l|}
\hline
周波数{[}kHz{]} & Rの電圧 {[}V{]} \\ \hline \hline
0.5          & 0.3119       \\ \hline
0.6          & 0.3564       \\ \hline
0.8          & 0.4625       \\ \hline
1            & 0.573        \\ \hline
2            & 0.845        \\ \hline
3            & 0.947        \\ \hline
3.5          & 0.976        \\ \hline
4            & 0.978        \\ \hline
4.5          & 0.975        \\ \hline
5            & 0.970        \\ \hline
6            & 0.938        \\ \hline
8            & 0.866        \\ \hline
10           & 0.775        \\ \hline
20           & 0.481        \\ \hline
30           & 0.33         \\ \hline
40           & 0.257        \\ \hline
50           & 0.0248       \\ \hline
60           & 0.1706       \\ \hline
80           & 0.1235       \\ \hline
100          & 0.0908       \\ \hline
\end{tabular}
\end{table}

\subsection{実験2の結果}
実験2の結果を表\ref{cal:result2}に示す。
\begin{table}[!h]
\centering
\caption{実験2の結果}
\label{cal:result2}
\begin{tabular}{|l|l|}
\hline
コンデンサ{[}pF{]} & Rの電圧 {[}V{]} \\ \hline
1             & 0.0819       \\ \hline
2             & 0.1942       \\ \hline
3             & 0.3483       \\ \hline
4             & 0.549        \\ \hline
5             & 0.768        \\ \hline
6             & 0.914        \\ \hline
7             & 0.947        \\ \hline
8             & 0.91         \\ \hline
9             & 0.852        \\ \hline
10            & 0.801        \\ \hline
20            & 0.575        \\ \hline
30            & 0.517        \\ \hline
40            & 0.492        \\ \hline
50            & 0.477        \\ \hline
60            & 0.4661       \\ \hline
70            & 0.4614       \\ \hline
80            & 0.4512       \\ \hline
90            & 0.4532       \\ \hline
100           & 0.4511       \\ \hline
150           & 0.4406       \\ \hline
200           & 0.4372       \\ \hline
\end{tabular}
\end{table}
\subsection{実験3の結果}
実験3の結果を表\ref{cal:result3}に示す。
\begin{table}[!h]
\centering
\caption{実験3の結果}
\label{cal:result3}
\begin{tabular}{|l|}
\hline
コイルの抵抗分 {[}Ω{]}            \\ \hline
\multicolumn{1}{|r|}{49.9} \\ \hline
\end{tabular}
\end{table}
\clearpage

\section{結果の整理}
\subsection{周波数特性}
以下の式より周波数に対する電流特性を図\ref{fig:f-I}に示す。
\begin{align}
I&=\frac{e_{R}}{R}
\end{align}

\begin{figure}[!h]
\center{
\includegraphics[clip, width=0.7\columnwidth]{img/f-I.eps}
\caption{周波数に対する電流特性}
\label{fig:f-I}
}
\end{figure}

また以下の式より周波数に対するインピーダンス特性を図\ref{fig:f-z}に示す。
\begin{align}
Z&=\frac{e}{I}\\
\end{align}

\begin{figure}[H]
\center{
\includegraphics[clip, width=0.7\columnwidth]{img/f-z.eps}
\caption{周波数に対するインピーダンス特性}
\label{fig:f-z}
}
\end{figure}

ここで共振周波数$f_{0}$について考える。
図\ref{fig:f-I}より電流が最大になる周波数が最大周波数と考えられるので、
\begin{align}
f_{0}&=4000\ \si{Hz}
\end{align}

これよりLの値が求められる。
\begin{align}{
&2\pi f_{0}L=\frac{1}{2\pi f_{0}C}\nonumber\\
&L=33.7\ \si{mH}\nonumber
}\end{align}

これらの値と以下の式を用いて周波数に対する位相特性を図\ref{fig:f-theta}に示す。
\begin{align}
\theta&=\tan^{-1}\frac{\omega L-\frac{1}{\omega C}}{R+R_{L}}
\end{align}

\begin{figure}[H]
\center{
\includegraphics[clip, width=0.7\columnwidth]{img/f-theta.eps}
\caption{周波数に対する位相特性}
\label{fig:f-theta}
}
\end{figure}

Lの値が分かったことにより電流の計算値$I_{math}$が求められる。
以下の式より電流の値を求めた上で周波数に対する電流特性を図\ref{fig:f-Imath}に示す。

\begin{align}
I_{math}&=\frac{V}{\sqrt{(R+r_{L})^2+(2\pi f_{0}L-\frac{1}{2\pi f_{0}})^2}}
\end{align}

\begin{figure}[H]
\center{
\includegraphics[clip, width=0.7\columnwidth]{img/f-Imath.eps}
\caption{周波数に対する電流特性(計算によって求めた値を使用)}
\label{fig:f-Imath}
}
\end{figure}

また$f<f_{0}$, $f=f_{0}$, $f>f_{0}$のときについて回路の状態をベクトル図で表現する。

\begin{figure}[H]
\center{
\subfigure[f<$f_{0}$の場合]{
\includegraphics[clip, width=0.25\columnwidth]{img/ff0.eps}
}
\subfigure[f=$f_{0}$の場合]{
\includegraphics[clip, width=0.25\columnwidth]{img/f0.eps}
}
\subfigure[f>$f_{0}$の場合]{
\includegraphics[clip, width=0.25\columnwidth]{img/f0f.eps}
}}
\caption{ベクトル図}
\end{figure}

また回路の尖鋭度を求める。
実験1の尖鋭度を$Q_{1}$、$Q_{2}$とする。
\begin{align}
Q_{1}&=\frac{1}{2\pi f_{0}RC}\nonumber\\
&=0.847\\
Q_{2}&=0.423\
\end{align}
\subsection{C依存性}
実験3より静電容量に対する電流特性を図\ref{fig:c-I}に示す。
\begin{figure}[H]
\center{
\includegraphics[clip, width=0.7\columnwidth]{img/c-I.eps}
\caption{静電容量に対する電流特性}
\label{fig:c-I}
}\end{figure}
%
\section{検討}
\subsection{周波数特性}
はじめに実験1(2)の5 kHzのときRにかかる電圧が極端に落ちることについて考える。
この値は前後と比べてみても明らかにおかしい。
原因としては読み間違いやきちんと端子が接触しなかったことによるものが考えられる。
それ以外の値は理想的なグラフ上にあるので今回この値は無視することにする。

まず図\ref{fig:f-I}について考える。
グラフの概形として理想的な形にかなり近い。
また実験1(1)と実験1(2)の最大値はそれぞれ0.946 mA、0.489 mAである。
これらの値は図\ref{fig:2a}より共振周波数の時に$\frac{E}{R}$となるはずである。
その理論値はそれぞれ1 mA、0.5 mAとなり非常に近い値だということが分かる。

次に図\ref{fig:f-z}について考える。
これも同様にグラフの概形はかなり理想的な形に近い。
図\ref{fig:2b}より最小値共振周波数の時にRとなるはずである。
そこで調べてみると、実験1(1)と実験1(2)はそれぞれ1057 $\Omega$、2044 $\Omega$である。
かなり近い値ではあるが些細な誤差があるのはコイルの抵抗分によるものと考えられる。
そこで実験3によるとコイルの抵抗分は49.9 $\Omega$である。
その値を先ほどのものから引くと1007 $\Omega$、1994.1 $\Omega$となりほぼ等しい値となった。
抵抗自体の誤差もあるのでかなり精度のよい結果である。

次に図\ref{fig:f-theta}について考える。
グラフの概形か理想的なものである。
位相は共振周波数の時に0に、周波数が小さくなれば$-\frac{\pi}{2}$、大きくなれば$\frac{\pi}{2}$に近くなる。
共振周波数の時に実験1(1)、実験1(2)それぞれ、-0.0173 [rad]、0 [rad]と後者については完全に0になった。
前者に関しても値はかなり小さく精度がよい。
周波数が小さいとき、前者は-1.40 [rad]、-1.27 [rad]と$-\frac{\pi}{2}=-1.57$とは若干誤差が現れた。
しかし図からまだ傾きが0となっていないので周波数をもっと小さくすることでさらに理想値に近づくことが予想される。
周波数が大きいとき、前者は1.52 [rad]、1.47 [rad]であり周波数が小さいときに比べ誤差は小さくなった。
しかし前者はともかく後者については少し誤差が大きい。
これは周波数が小さい時にも言えることで後者の実験の方が位相の変化が緩慢になっている。
その理由は後に説明する尖鋭度によるものである。

次に図\ref{fig:f-Imath}について考える。
図\ref{fig:f-I}では抵抗値とそれにかかる電圧のみで電流を求めていた。
それに対しコンデンサや共振周波数などより多くのデータを用いて求めている。
少ないデータから求めた時それに含まれる誤差が計算結果に大きく影響を与えてしまう。
しかし多くのデータを用いることで誤差を減少させることができる。
つまりこのグラフの方が真値に近い。
そこで改めて見てみる。
図\ref{fig:f-I}に比べグラフの対称性が上がっているのが分かる。
また最大値に関しても0.952 [mA]、0.488 [mA]と精度よく測定できている。

次にベクトル図について考える。
周波数が小さいければ小さいほど容量性負荷の大きさは大きくなる。
よって$f<f_{0}$のときVの値はx成分が$V_{R}$、y成分が$V_{C}-V_{L}$になる。
また逆に周波数が大きければ大きいとき容量性負荷の大きさは小さくなる。
よって$f>f_{0}$のときVの値はx成分が$V_{R}$、y成分が$V_{L}-V_{C}$になる。
最後に$f=f{0}$のとき$V_{L}=V_{C}$になるので打ち消されて$V=V_{R}$となる。

尖鋭度については後の調査事項の章にて説明する。

\subsection{C依存性}
図\ref{fig:c-I}より考える。
まず、11 kHzで共振する静電容量の値$C_{0}$とそのときの電流の値$I_{0}$を求める。
\begin{align}
C&=\frac{1}{(2\pi \times 11\times 10^{3})^2\times 33.7\times 10^{-3}}\\
&=0.00621\ \si{\micro F}\\
I_{0}&=\frac{V}{\sqrt{(R+r_{L})^2+(\omega L-\frac{1}{\omega C})^2}}\\
&=0.952\ \si{mA}
\end{align}
また図の最大値の時の静電容量は0.007 $\si{\micro F}$、電流値は0.947 mAである。
静電容量に関しては少し誤差が目立った。
その原因としてコンデンサ自体の誤差が考えられる。
またその他の原因としては、Lの値は既知の値ではなく計算で求めた値なので誤差が含まれている可能性がある。
ただ電流の値はかなり近い値になっている。
さらにLの値を用いた先の計算でも誤差がかなり小さく抑えられているので、Cの値の誤差としては前者の理由が大きいと考えられる。

また同じこの図では共振している時に対して線対称になっていない。
これの原因について考える。
まず静電容量が小さいときインピーダンスは非常に大きい値となるため電流は小さくなる。
また逆の場合について、静電容量のリアクタンスは非常に小さい値となる。
よって全体のインピーダンスの大きさは$\sqrt{(R+r_{L})^2+(2\pi \times 11\times 10^{-3})^2\times L}=2554\ \Omega$となる。
それにより電流の値は0.391 mAに近づく。
このことは図からも確認できるだろう。

\section{調査事項}
\subsection{共振回路の利用場面}

\subsection{尖鋭度について}
%
\begin{thebibliography}{99}
\bibitem{ref:yasashi} abc
\end{thebibliography}
\end{document}